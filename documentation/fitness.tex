\documentclass[12pt]{article}
\usepackage{amsmath}

\title{Reparm: Genesis}
\author{Dustin Tracy}

\begin{document}

\maketitle

\section{Overview}
Each function analyzes only a parametergroup containing all the structures but only one set of parameters.


\section{Energy Fitness}
All the energies of each structure is contained in a vector $\mathbf{E}$.
The differences between the energies of each structure is then inserted into a new vector
\begin{equation}
  \label{eq:energy_differences}
  \Delta E_{i} = E_{i+1} - E_{i}
\end{equation}
This calculation is done for both AM1 and the HLT creating vectors $\mathbf{\Delta E_{AM1}}$ and $\mathbf{\Delta E_{HLT}}$.
The distance between the two vectors is returned.


\section{Dipole Average Fitness}
We start with a 2 dimensional array, with the rows representing each structure, and the columns representing the magnitude of the dipole moment in coordinate $x_i$.
\[
\begin{bmatrix}
  x_{1} & y_{1} & z_{1} \\
  x_{2} & y_{2} & z_{2} \\
  \hdotsfor{3}   \\
  x_n & y_n & z_n
\end{bmatrix}
\]
We average the values in each row to get the average dipole moment.
Doing this for both AM1 and HLT yields

\begin{equation}
  \mathbf{D_{AM1|HLT}} = \left[x, y, z\right]
\end{equation}

We then return the distance between $\mathbf{D_{AM1}}$ and $\mathbf{D_{HLT}}$.

\section{Dipole Difference Fitness}
For each level of theory, we once again start with the 2 dimensional array
\[
\begin{bmatrix}
  x_{1} & y_{1} & z_{1} \\
  x_{2} & y_{2} & z_{2} \\
  \hdotsfor{3}   \\
  x_n & y_n & z_n
\end{bmatrix}
\]
We can represent represent each dipole by a 3 component vector, leading to
\[
  \begin{bmatrix}
   \mathbf{d_1} \\
   \mathbf{d_2} \\
   \dots        \\
   \mathbf{d_n} \\
  \end{bmatrix}
\]
A matrix of the differences can now be created
\begin{equation}
  \mathbf{\Delta D}=
  \begin{bmatrix}
    0 & | \mathbf{d_2} - \mathbf{d_1} | & | \mathbf{d_3} - \mathbf{d_1} |  & \dots & | \mathbf{d_n} - \mathbf{d_1} | \\
    | \mathbf{d_1} - \mathbf{d_2} | & 0 & | \mathbf{d_3} - \mathbf{d_2} |  & \dots & | \mathbf{d_n} - \mathbf{d_2} | \\
    \hdotsfor{5} \\
    | \mathbf{d_1} - \mathbf{d_n} | &  | \mathbf{d_2} - \mathbf{d_n} | & | \mathbf{d_3} - \mathbf{d_2} |  & \dots & 0 
  \end{bmatrix}
\end{equation}
The distance is between $\mathbf{\Delta D_{AM1}}$ and $\mathbf{\Delta D_{HLT}}$ is returned.


\end{document}
