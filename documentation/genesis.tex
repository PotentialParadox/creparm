\documentclass[12pt]{article}
\usepackage{amsmath}

\title{Reparm: Genesis}
\author{Dustin Tracy}

\begin{document}

\maketitle

\section{Geometry array creation}
The expected energy of an atom is
\begin{equation}
  \label{eq:energy_atom}
  \left<E_{atom}\right>=\frac{3}{2}K_{b}T
\end{equation}
where $K_b$ is the boltzman constant $1.38064853E^{-23}m^2kgS^{-2}K^{-1}$ and $T$ is temperature which is defaulted to $298K$ the default for Gaussian 09.
The total energy of the molecule is then.
\begin{equation}
  \label{eq:energy_molecule}
 \left<E_{molecule}\right>=\sum_{i=atom}^{N}E_{atom}
\end{equation}
where N is the number of atoms.
Of course this is the expected value.
We therefore need to use a random number from a normal distribution to modify it. 
The variance of molecule energy is
\begin{equation}
  \label{eq:energy_variance}
  \sigma^2=\frac{2}{3N}\left<E_{molecule}\right>
\end{equation}
Leading to the molecule energy
\begin{equation}
  \label{eq:adj_molecule_energy}
  E_{molecule}= \left<E_{molecule}\right> + R
\end{equation}
where $R$ is the random value generated from a normal distribution with aforementioned variance.
We then distribute this energy among the normal modes using a uniform random number generator, giving us a new vector $E_{mode}$.
Now, normal modes are represented by harmonic oscillators, so 
\begin{align}
  E_{mode}&=\frac{1}{2}kx^2\\
  x&=\sqrt{\frac{2E}{k}}
\end{align}
where $k$ is the spring constant given  and x is the displacement from equilibrium.
Now these diplacements, $\mathbf{x}$, can be made in any direction, so we once again apply a random number generator to flip the sign of some some of these modes.
We now have our displacement vector $\mathbf{x}$.
For simplification of calculations, we make this a diagonal matrix $\mathbf{X}$.
Our normal modes are extracted from gaussian 09 and placed into a matrix $\mathbf{M}$ with the columns representing the coordinates for each atom, $X_{atom1}$, $Y_{atom1}$, $Z_{atom1}$, $X_{atom2}$, etc., and the rows representing the normal modes.
Our diplacement matrix is thus,
\begin{equation}
  \label{eq:displacement}
 \mathbf{D}=\mathbf{XM} 
\end{equation}
When we sum the rows we obtain the displacement vector $\mathbf{d}$, which is then applied to the optimized coordinates.
\end{document}
